\hypertarget{1}{}

\chapter{Introduction}

\rhead{Introduction}
\lhead{Chapter 1}

\vspace{-1.6cm}

% Gray Line
\begingroup
\color{gray}
\par\noindent\rule{\textwidth}{0.4pt}
\endgroup

\noindent{This chapter presents the motivation, objectives, general methodology, and contributions of this dissertation, as well as the overall document structure.}

% ------------------------------> MOTIVATION

\section{Motivation}

% ------------------------------> DOCUMENT STRUCTURE

\section{Document Structure}

Additionally to the present introductory chapter, this document is structured in four chapters as follows:

\begin{itemize}
   \item \textbf{Chapter \hyperlink{2}{2}} (Related Work) introduces the basic concepts and resources that support RE techniques, namely, Natural Language Processing (NLP), text mining primary tasks, initial approaches for RE, distant supervision for RE, neural networks for RE, and evaluation measures.
   \item \textbf{Chapter \hyperlink{3}{3}} (A Silver Standard Corpus of Phenotype-Gene Relations) presents the work developed to create a silver standard corpus of human phenotype-gene relations, including methods, evaluation, results and discussion.
   \item \textbf{Chapter \hyperlink{4}{4}} (Extracting Phenotype-Gene Relations) presents the system modules developed (distantly supervised multi-instance and deep learning modules) to accommodate human phenotype-gene RE, with methods, evaluation, results and discussion, for each module, and a detailed comparison between the two.
   \item \textbf{Chapter \hyperlink{5}{5}} (Conclusion) discusses the main conclusions of this work, and indicates some directions for future work.
\end{itemize}